\documentclass{mcmthesis}

%------------begin preamble---------------
\usepackage{newtxtext}%\usepackage{palatino}
\usepackage{lipsum}

\def\TeamNumber{2112243}
\def\Problem{C}

\mcmsetup{
    CTeX = false, % 写中文文档时,设置为 true
    tcn = \TeamNumber,
    problem = \Problem,
    sheet = true,
    titleinsheet = true,
    abstract = true,
    keywordsinsheet = true,
    titlepage = false % 去掉重复的summary page
}

\title{\LaTeX{} Template for MCM Thesis of Team \#\TeamNumber}
\author{}
\date{}
%------------end preamble---------------

\begin{document}

%------------begin summary sheet---------------
\begin{abstract}
Use this template to begin typing the first page (summary page) of your electronic report. This template uses a 12-point Times New Roman font. Submit your paper as an Adobe PDF electronic file (e.g. 1111111.pdf), typed in English, with a readable font of at least 12-point type.

Do not include the name of your school, advisor, or team members on this or any page.

Papers must be within the \textcolor[rgb]{1,0,0}{25 page} limit.

Be sure to change the control number and problem choice above.
You may delete these instructions as you begin to type your report here.

Follow us @COMAPMath on Twitter or COMAPCHINAOFFICIAL on Weibo for the most up to date contest information.

\begin{keywords}
keyword1; keyword2
\end{keywords}
\end{abstract}

\maketitle
%------------end summary sheet---------------

%------------begin table of contents---------------
\tableofcontents
\newpage
%------------end table of contents---------------


% Generate the Memorandum, if it's needed.
\memoto{\LaTeX{}studio}
\memofrom{Liam Huang}
\memosubject{Happy \TeX{}ing!}
\memodate{\today}
% \logo{\LARGE I'm pretending to be a LOGO!} % \logo command undefined
\begin{memo}[Memorandum]
  \lipsum[1-3]
\end{memo}


%============begin thesis (main part)===============
\section{Introduction}
\subsection{What's this all about? What's \LaTeX?}
\LaTeX\ is a document preparation system which uses the \TeX\
typesetting program. It enables you to produce
publication-quality documents with great accuracy and
consistency. \LaTeX\ works on any computer and produces
industry-standard PDF. It is available both in free (open-source)
and commercial implementations. \LaTeX\ can be used for any kind
of document, but it is especially suited to those with complex
structures, repetitive formatting, or notations like
mathematics. Install the software from
\url{www.tug.org/texlive/}.
\subsection{Creating and typesetting your document}

\subsection{Syntax (how to type \LaTeX\ commands --- these
  are the rules)}

\lipsum[3]
\begin{itemize}
\item the angular velocity of the bat,
\item the velocity of the ball, and
\item the position of impact along the bat.
\end{itemize}
\lipsum[4]
\emph{center of percussion} [Brody 1986], \lipsum[5]

\begin{Theorem} \label{thm:latex}
This is a theorem.
\end{Theorem}

\begin{Lemma} \label{thm:tex}
This is a lemma.
\end{Lemma}

\begin{proof}
The proof of theorem.
\end{proof}

\subsection{Other Assumptions}
\lipsum[6]
\begin{itemize}
\item an assumption
\item an assumption
\item an assumption
\item an assumption
\end{itemize}

\lipsum[7]

\section{Analysis of the Problem}
\begin{figure}[h]
    \small
    \centering
    \includegraphics[width=8cm]{example-image-a}
    \caption{The name of figure} \label{fig:aa}
\end{figure}

\lipsum[8] \eqref{aa}
\begin{equation}
a^2 \label{aa}
\end{equation}

\[
    \begin{pmatrix}{*{20}c}
    {a_{11} } & {a_{12} } & {a_{13} }  \\
    {a_{21} } & {a_{22} } & {a_{23} }  \\
    {a_{31} } & {a_{32} } & {a_{33} }  \\
    \end{pmatrix}
    = \frac{{Opposite}}{{Hypotenuse}}\cos ^{ - 1} \theta \arcsin \theta
\]

\lipsum[9]

\[
    p_{j}=\begin{cases} 0,&\text{if $j$ is odd}\\
    r!\,(-1)^{j/2},&\text{if $j$ is even}
    \end{cases}
\]

\lipsum[10]

\[
    \arcsin \theta  =
    \mathop{{\int\!\!\!\!\!\int\!\!\!\!\!\int}\mkern-31.2mu
    \bigodot}\limits_\varphi
    {\mathop {\lim }\limits_{x \to \infty } \frac{{n!}}{{r!\left( {n - r}
    \right)!}}} \eqno (1)
\]

\section{Calculating and Simplifying the Model  }
\lipsum[11]

\section{The Model Results}
\lipsum[6]

\section{Validating the Model}
\lipsum[9]

\section{Conclusions}
\lipsum[6]

\section{A Summary}
\lipsum[6]

\section{Evaluate of the Mode}

\section{Strengths and weaknesses}
\lipsum[12]

\subsection{Strengths}
\begin{itemize}
\item \textbf{Applies widely}\\
This  system can be used for many types of airplanes, and it also
solves the interference during  the procedure of the boarding
airplane,as described above we can get to the  optimization
boarding time.We also know that all the service is automate.
\item \textbf{Improve the quality of the airport service}\\
Balancing the cost of the cost and the benefit, it will bring in
more convenient  for airport and passengers.It also saves many
human resources for the airline.
\item {\large\bf{This is a citation TEST.}} \cite{knuth1984tex} \cite{lamport1986latex} \cite{latexstudio.net}
\end{itemize}
%============end thesis (main part)===============

% user bibtex instead of this template
\bibliographystyle{unsrt} % reference按引用先后排序
\bibliography{refer} % build reference from file "refer.bib"


% \begin{thebibliography}{99}
% \bibitem{1} D.~E. KNUTH   The \TeX{}book  the American
% Mathematical Society and Addison-Wesley
% Publishing Company , 1984-1986.
% \bibitem{2}Lamport, Leslie,  \LaTeX{}: `` A Document Preparation System '',
% Addison-Wesley Publishing Company, 1986.
% \bibitem{3}\url{https://www.latexstudio.net/}
% \end{thebibliography}


\begin{appendices}

\section{First appendix}

In addition, your report must include a letter to the Chief Financial Officer (CFO) of the Goodgrant Foundation, Mr. Alpha Chiang, that describes the optimal investment strategy, your modeling approach and major results, and a brief discussion of your proposed concept of a return-on-investment (ROI). This letter should be no more than two pages in length.

\begin{letter}{Dear, Mr. Alpha Chiang}

\lipsum[1-2]

\vspace{\parskip}

Sincerely yours,

Your friends

\end{letter}
Here are simulation programmes we used in our model as follow.\\

\textbf{\textcolor[rgb]{0.98,0.00,0.00}{Input matlab source:}}
\lstinputlisting[language=Matlab]{./code/mcmthesis-matlab1.m}

\section{Second appendix}

some more text \textcolor[rgb]{0.98,0.00,0.00}{\textbf{Input C++ source:}}
\lstinputlisting[language=C++]{./code/mcmthesis-sudoku.cpp}

\end{appendices}
\end{document}
